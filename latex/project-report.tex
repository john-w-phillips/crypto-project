\documentclass[conference]{IEEEtran}
\IEEEoverridecommandlockouts
% The preceding line is only needed to identify funding in the first footnote. If that is unneeded, please comment it out.
\usepackage{cite}
\usepackage{amsmath,amssymb,amsfonts}
\usepackage{algorithmic}
\usepackage{graphicx}
\usepackage{textcomp}
\usepackage{xcolor}
\def\BibTeX{{\rm B\kern-.05em{\sc i\kern-.025em b}\kern-.08em
    T\kern-.1667em\lower.7ex\hbox{E}\kern-.125emX}}
\begin{document}

\title{Security of OpenSSL Tool}

\author{\IEEEauthorblockN{John Phillips}
\textit{johphill@mines.edu}\\
\and
\IEEEauthorblockN{2\textsuperscript{nd} Abhaya Shrestha}
\IEEEauthorblockA{\textit{ashrestha@mines.edu}}
\and
\IEEEauthorblockN{3\textsuperscript{rd} Joe Granmoe}
\IEEEauthorblockA{\textit{jgranmoe@mines.edu}}
\and
\IEEEauthorblockN{4\textsuperscript{th} Patrick Curran}
\IEEEauthorblockA{\textit{pcurran@mines.edu}}
\and
\IEEEauthorblockN{5\textsuperscript{th} Collin McDade}
\IEEEauthorblockA{\textit{collinmcdade@mines.edu}}
\and
\IEEEauthorblockN{6\textsuperscript{th} Noor Malik}
\IEEEauthorblockA{\textit{nmalik@mines.edu}}
}

\maketitle

\begin{abstract}
There have been small but catastrophic modifications to openssl in the
past\cite{1}\cite{2} that drastically reduced the potential number of
keys that could be generated during key generation in the debian
distribution. We've reproduced one from 2008 to get hands-on
experience with how these kinds of accidents happen, and in the hopes
of understanding how to avoid such a impactful mistake.
\end{abstract}

\begin{IEEEkeywords}
OpenSSL, security vulnerability, debian
\end{IEEEkeywords}

\section{Introduction}
\section{Background}
In 2008 a debian developer made a patch to openssl to fix valgrind
warnings\cite{2}\cite{3}. OpenSSL has a flexible system for generating
random numbers that uses an interface-like approach with functions
prefixed with \verb|RAND_*| delegating to a particular
implementation. Since this API was designed for cryptographic
applications, the random bytes generated need to not just satisfy
particular statistical properties but be truly unpredictable. In order
to acheive this property, this particular random number generator
allows the user to add 'entropy' by feeding it buffers full of random
bytes. These bytes could be from \verb|/dev/urandom|, the user, or
somewhere else. When adding entropy in this way to the message
digest-based random API, an internal hash state is update each time
with the random bytes given by the user. The hash state then can be
used to get random bytes back out of the system. 

Now this particular implementation in \verb|md_rand.c|,
\emph{deliberately} would `over-read' the user-provided buffer. The
apparent thinking behind this was that if the code read uninitialized
memory, this couldn't hurt things, and could only possibly help by
potentially adding randomness.

This behavior occurred in two places in \verb|md_rand.c|. Both of
those places were actually critical to the entire random number
generator. Because of the reading of uninitialized memory, any time a
user of the ssl library did almost anything -- like generate a key,
for example -- they would get disturbing warnings if they ran their
program in the popular \verb|valgrind| tool.

This was the source of a complaint that first came to the debian bug
tracker\cite{2}. The debian developer quickly spotted the culprit code
but didn't know what it did. Observing that commenting it out didn't
cause anything to crash or obviously function incorrectly, he asked on
the openssl development mailing list if it was okay to remove the two
lines of code -- and got the go-ahead.

Most linux distributions maintain small set of `patches' for each
package they distribute. These patches normally are designed to fix
critical bugs that haven't been fixed upstream or are in the process
of being fixed,





\begin{thebibliography}{00}

\bibitem{1} https://isotoma.com/blog/2008/05/14/debians-openssl-disaster/
\bibitem{2} https://bugs.debian.org/cgi-bin/bugreport.cgi?bug=363516
\bibitem{3} https://research.swtch.com/openssl  
\end{thebibliography}

\end{document}
